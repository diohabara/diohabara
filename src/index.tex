\documentclass[10pt]{article}
\usepackage[T1]{fontenc}
\usepackage{newpxtext, newpxmath}
\usepackage{dirtytalk}
\usepackage[margin=15mm]{geometry}
\usepackage{fancyhdr}
\usepackage{cmbright} % font
\usepackage{hyperref}
% reduce space
\usepackage{titlesec}
\titlespacing* % the star = don't indent first paragraph after
    {\section} % which command you want to set the spacing for
    {0pt} % spacing to the left of heading
    {1ex} % spacing before the heading
    {1ex} % spacing after the heading
\titlespacing*
    {\subsection}
    {0pt}
    {1ex}
    {1ex}
\titlespacing*
    {\subsubsection}
    {0pt}
    {1ex}
    {1ex}
\usepackage{multicol} % multiple column listing
% reduce space between and above listing
\setlength{\columnsep}{10pt}
\setlength{\multicolsep}{0pt}
\usepackage{enumitem}
% configuration
% hyperlink coloring
\hypersetup{
    colorlinks=true,
    filecolor=magenta,
    urlcolor=cyan,
}
\urlstyle{same}
% set header
\pagestyle{fancy}
\lhead{
  Curriculum Vitae/Resume
  , Takemaru Kadoi
  , \href{mailto:diohabara@gmail.com}{diohabara@gmail.com}
  , \href{https://github.com/diohabara/}{GitHub Link}
  , \href{https://www.linkedin.com/in/takemaru-kadoi}{LinkedIn Link}
  , \href{https://scrapbox.io/jampon/}{Scrapbox Link}
}
\usepackage{amsmath}
\usepackage{amssymb}

% macro

% content
\title{\vspace{-1cm}Curriculum Vitae/Resume}
\author{Takemaru Kadoi}
\date{}

\begin{document}

\section*{Education}
  \subsection*{\underline{The University of Texas at Dallas}, 08/2022 - 05/2024(expected) \hfill Richardson, Texas}
    Master of Science in \textbf{Computer Science}
    \subsubsection*{Relevant Courses}
    \begin{multicols}{2}
      \setlist{nolistsep}
      \begin{itemize}[noitemsep]
        \item Advanced Computer Networks(currently enrolled)
        \item Discrete Structures(currently enrolled)
        \item System Security and Binary Analysis(currently enrolled)
      \end{itemize}
    \end{multicols}

  \subsection*{\underline{The University of Tokyo}, 04/2017 -  03/2022 \hfill Tokyo, Japan}
    Bachelor of Science in \textbf{Electrical Engineering \& Computer Science}
    \\
    Studied computer science at Stanford University as an exchange student from 06/2019-08/2019
    \\
    Wrote a bachelor thesis, "Type- and Sequential Effect-Guided Programming by Example," supervised by Prof. Masahiro FUJITA
    \subsubsection*{Relevant Courses}
    \begin{multicols}{2}
      \setlist{nolistsep}
      \begin{itemize}[noitemsep]
        \item Algorithms
        \item Artificial Intelligence
        \item Basics In Mathematics I/II
        \item Computer Networks
        \item Computer Software I/II
        \item Digital Circuits
        \item Distributed Systems
        \item Electronic Information Processing Devices
        \item Fundamental Exercise on Programming
        \item Hardware Design
        \item Image Media Technologies
        \item Information Security
        \item Information Theory
        \item Introduction to Algorithms
        \item Introduction to Python Programming
        \item Introduction to Statistics
        \item Language Processing I
        \item Operating Systems
        \item Probability and Statistics
        \item Programming Languages
        \item Statistical Machine Learning
        \item Theory of Computation
      \end{itemize}
    \end{multicols}

\section*{Work Experience}
  \subsection*{\underline{Software Engineer Intern}, Indeed, 06/2022 - 08/2022 \hfill Tokyo, Japan}
    Created a proof-of-concept data pipeline for its data infrastructure so that the data team and other teams could easily collaborate with the same API in different programming languages.
    The implementation allowed more people to analyze data in the company.
  \subsection*{\underline{Software Engineer Intern}, Hatena, 02/2022 - 06/2022 \hfill Tokyo, Japan}
    Migrated CI/CD pipelines from Jenkins to GitHub Actions and sped them up, resulting in better production quality by improving the development cycle and increasing the number of reviews and deployments.
  \subsection*{\underline{Software Engineer Intern}, Q-Squared Technologies, 06/2021 - 02/2022 \hfill Tokyo, Japan}
    Created a proof-of-concept for network acceleration.
    My work contributed to the decision about what the company would select as a network library for the company.
  \subsection*{\underline{Software Engineer Intern}, Mercari, 08/2020 - 09/20209 \hfill Tokyo, Japan}
    Created a static analysis tool for the Go programming language.
  \subsection*{\underline{Software Engineer Intern}, FLYWHEEL, 07/2020 - 08/2020 \hfill Tokyo, Japan}
    Introduced the speed layer of the lambda architecture in the corporate data platform.
    The new layer potentially expanded the corporate business because streaming data analytics was useful for real-time analysis.
  \subsection*{\underline{Software Engineer Intern}, Wantedly, 08/2019 - 09/2019 \hfill Tokyo, Japan}
    Improved existing functions of Wantedly People, an HR application, and added new functionalities.
    Those new functionalities were related to user management and enabled the company to find malicious users.

\section*{Research Experience}
  \subsection*{"Type- and Sequential Effect-Guided Programming by Example," supervised by Prof. Masahiro FUJITA}
    Wrote the bachelor thesis on program synthesis.
    Implemented an ML-like target language in OCaml.
    Its synthesizer used the language's type and effect system and given examples.

\section*{Projects}
  \subsection*{A toy C compiler written in C itself: \href{https://github.com/diohabara/ccc}{ccc}}
    \textbf{By myself}
    \\
    Implemented a toy C compiler in C with a lexer, parser, and code generator.
    It generates x86-64 code.
  \subsection*{The top 3 in ICCAD 2021 CAD Contest}
    \textbf{In teams of 6}
    \\
    Teamed up with other lab members and worked on the circuit generation section in B4.
    Wrote evaluations of methods based on our criteria, actual statistics, and goal.
    My main task was testing what combination of methods we should choose.
  \subsection*{BrainF**k interpreter on Raspberry Pi written in Python: \href{https://github.com/diohabara/muscle_fuck}{muscle\_f**k}}
    \textbf{By myself}
    \\
    Implemented a Raspberry Pi application to enable people to get exercise while programming in B4
  \subsection*{Implementation of TCP/IP protocol with KLab}
    \textbf{By myself}
    \\
    Implemented TCP/IP protocol in C with mentors and other participants as an internship project in B4.
  \subsection*{Implementation of bookmark functionalities to Firefox}
    \textbf{In teams of 2}
    \\
    Added additional bookmark functionalities to Firefox with another member from the same department as a school project in B3.
    Built Firefox and modified the source code.

\section*{Technical Skills}
  \subsection*{Programming Languages}
    = \{
      C/C++,
      Common Lisp/Scheme/Racket,
      Coq,
      Go,
      HCL,
      Haskell,
      Java/Scala/Kotlin,
      JavaScript/TypeScript,
      Nix Language,
      OCaml,
      Python3,
      Ruby,
      Rust,
      Shell Script,
      Verilog/System Verilog,
      $\varnothing$
    \}
  \subsection*{Cloud Technologies}
    = \{
      AWS(
        EC2,
        S3,
        Kinesis
      ),
      GitHub Actions,
      $\varnothing$
    \}
  \subsection*{Others}
    = \{
      Data Plane Development Kit(DPDK),
      Docker,
      Flask,
      Flink,
      Kafka,
      Nix Ecosystem,
      OCamllex/Menhir,
      Pandas,
      PyTorch,
      React,
      Ruby on Rails,
      Spark,
      Terraform,
      scikit-learn,
      $\varnothing$
    \}

\section*{Interests}
  = \{
    Combination of Static and Dynamic Analyses,
    Database Management System,
    Formal Verification,
    Program Synthesis,
    Programming Languages,
    System Programming,
    Testing,
    Type Theory,
    $\varnothing$
  \}
  ||
  \{
    Browser,
    Cloud,
    Education,
    FOSS,
    Finance,
    $\varnothing$
  \}

\end{document}