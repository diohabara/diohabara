\documentclass[12pt]{article}
\usepackage{hyperref}
\usepackage[T1]{fontenc}
\usepackage{newpxtext, newpxmath}
\usepackage{dirtytalk}
\usepackage[margin=15mm]{geometry}
\hypersetup{
    colorlinks=true,
    filecolor=magenta,
    urlcolor=cyan,
}
\urlstyle{same}

\title{\vspace{-1cm}Curriculum Vitae/Resume}
\author{Takemaru Kadoi}
\date{}

\usepackage{fancyhdr}
\pagestyle{fancy}
  \lhead{
    Curriculum Vitae/Resume,
    Takemaru Kadoi,
  }

\begin{document}

% \maketitle

\section*{Contacts}
  \{
    \href{https://github.com/diohabara/}{GitHub}
    , \href{https://www.linkedin.com/in/takemaru-kadoi}{LinkedIn}
    , \href{mailto:diohabara@gmail.com}{diohabara@gmail.com}
  \}

\section*{Educations}
  \subsection*{The University of Texas at Dallas, August 2022 - Present}
    Graduate Student in Computer Science, Computer Science Department
  \subsection*{The University of Tokyo, April 2017 - March 2022}
    Bachelor in EECS(Electrical Electronics and Computer Science), Department of Engineering \\
    Wrote a senior thesis, "Type- and Sequential Effect-Guided Programming by Example," supervised by \href{https://www.cad.t.u-tokyo.ac.jp/en/}{Masahiro FUJITA}
  \subsection*{Stanford University, June 2019 - September 2019}
    Exchange Student, Computer Science

\section*{Employments}
  \subsection*{\href{https://www.indeed.com/about}{Indeed}: Software Engineer Intern, July 2022 - August 2022}
    I created a PoC data pipeline for its data infrastructure so that the data team and other teams can easily collaborate with the same API.
  \subsection*{\href{https://hatenacorp.jp/}{Hatena}: Software Engineer Intern, February 2022 - May 2022}
    I improved the developer experience by migrating from Jenkins to GitHub Actions and speeding up CI/CD.
  \subsection*{\href{https://q-squared.jp}{Q-Squared}: Software Engineer Intern, June 2021 - February 2022}
    I made a proof of concept for network acceleration.
    \\
    \href{https://flossy-era-126.notion.site/DPDK-Potential-in-Finance-840b0d289273495192eb04d97d268eeb}{Blog post in Japanese}
  \subsection*{\href{https://about.mercari.com/en}{Mercari}: Software Engineer Intern, August 2020 - September 2020}
    I made a static analysis tool for the Go programming language.
  \subsection*{\href{https://www.flywheel.jp}{FLYWHEEL}: Software Engineer Intern, July 2020 - August 2020}
    I made a real-time layer of the data platform.
    \\
    \href{https://www.flywheel.jp/topics/20200917}{Blog post in Japanese}
  \subsection*{\href{https://wantedlyinc.com/ja}{Wantedly}: Software Engineer Intern, August 2019 - September 2019}
    I improved existing functions of \href{https://people.wantedly.com/}{Wantedly People}, an HR application, and added new functionalities.

\section*{Projects}
  \subsection*{The top 3 in \href{http://iccad-contest.org/2021/}{ICCAD 2021 CAD Contest}}
    \say{The CAD Contest at ICCAD is a challenging, multi-month research and development competition focusing on advanced, real-world problems in the field of Electronic Design Automation (EDA).} \\
    I worked on circuit generation with other lab members in B4.
  \subsection*{Implementation of TCP/IP protocol with \href{https://www.klab.com/en/}{KLab}}
    I implemented TCP/IP protocol in C with mentors and other participants as an internship project in B4.
  \subsection*{BrainF**k interpreter on Raspberry Pi: \href{https://github.com/diohabara/muscle_fuck}{muscle\_f**k}}
    I created a Raspberry Pi app to enable people to get exercise while programming in B4
  \subsection*{Addition of some functionalities to \href{https://slides.com/diohabara/deck}{Firefox}}
    I added additional bookmark properties to Firefox as a school project in B3.

\section*{Hobby repositories}
  \subsection*{A toy C compiler: \href{https://github.com/diohabara/ccc}{ccc}}
    I wrote a toy C compiler with an original lexer, parser, and code generator.

\end{document}
