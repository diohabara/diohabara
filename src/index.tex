\documentclass[10pt]{article}
\usepackage[T1]{fontenc}
\usepackage{newpxtext, newpxmath}
\usepackage{dirtytalk}
\usepackage[margin=15mm]{geometry}
\usepackage{fancyhdr}
\usepackage{cmbright} % font
\usepackage{hyperref}
% reduce space
\usepackage{titlesec}
\titlespacing* % the star = don't indent first paragraph after
    {\section} % which command you want to set the spacing for
    {0pt} % spacing to the left of heading
    {1ex} % spacing before the heading
    {1ex} % spacing after the heading
\titlespacing*
    {\subsection}
    {0pt}
    {1ex}
    {1ex}
\titlespacing*
    {\subsubsection}
    {0pt}
    {1ex}
    {1ex}
\usepackage{multicol} % multiple column listing
% reduce space between and above listing
\setlength{\columnsep}{10pt}
\setlength{\multicolsep}{0pt}
\usepackage{enumitem}
% configuration
% hyperlink coloring
\hypersetup{
    colorlinks=true,
    filecolor=magenta,
    urlcolor=cyan,
}
\urlstyle{same}
\usepackage{amsmath}
\usepackage{amssymb}
% font
\usepackage{fontawesome}

% macro

% set header
\pagestyle{fancy}
\lhead{
  % Curriculum Vitae/Resume
  Resume
  , Takemaru Kadoi
  , \faicon{envelope}
  \href{mailto:diohabara@gmail.com}{diohabara@gmail.com}
  , \faicon{github-alt}
  \href{https://github.com/diohabara/}{GitHub}
  , \faicon{linkedin} 
  \href{https://www.linkedin.com/in/takemaru-kadoi}{LinkedIn}
}

% content
\title{\vspace{-1cm}Curriculum Vitae/Resume}
\author{Takemaru Kadoi}
\date{}

\begin{document}

\section*{Education}
  \subsection*{\underline{The University of Texas at Dallas}, 08/2022 - 05/2024(expected) \hfill Richardson, Texas}
    Master of Science in \textbf{Computer Science}
    \\
    Researching on a security topic supervised by Dr. Kangkook Jee.

  \subsection*{\underline{The University of Tokyo}, 04/2017 -  03/2022 \hfill Tokyo, Japan}
    Bachelor of Science in \textbf{Electrical Engineering \& Computer Science}
    \\
    Studied computer science at Stanford University as an exchange student from 06/2019-08/2019
    \\
    Wrote a bachelor thesis, "Type- and Sequential Effect-Guided Programming by Example," supervised by Prof. Masahiro FUJITA

\section*{Work Experience}
  \subsection*{\underline{Software Engineer Intern}, Indeed, 06/2022 - 08/2022 \hfill Tokyo, Japan}
    Created a proof-of-concept data pipeline for its data infrastructure so that the data team and other teams could easily collaborate with the same API in different programming languages.
    The implementation allowed more people to analyze data in the company.
  \subsection*{\underline{Software Engineer Intern}, Hatena, 02/2022 - 06/2022 \hfill Tokyo, Japan}
    Migrated CI/CD pipelines from Jenkins to GitHub Actions and sped them up, resulting in better production quality by improving the development cycle and increasing the number of reviews and deployments.
  \subsection*{\underline{Software Engineer Intern}, FLYWHEEL, 07/2020 - 08/2020 \hfill Tokyo, Japan}
    Introduced the speed layer of the lambda architecture in the corporate data platform.
    The new layer potentially expanded the corporate business because streaming data analytics was useful for real-time analysis.

\section*{Research Experience}
  \subsection*{Undergraduate senior research supervised by Prof. Masahiro FUJITA}
    Wrote a bachelor thesis on program synthesis.
    Implemented an ML-like target language in OCaml.
    Its synthesizer utilized the language's type-, effect-system and given examples.

\section*{Projects}
  \begin{multicols}{2}
  \subsection*{Personal Project}
    \subsubsection*{BrainF**k interpreter on Raspberry Pi written in Python}
      \faicon{github} \href{https://github.com/diohabara/muscleFuck}{https://github.com/diohabara/muscleFuck}
      \\Implemented a Raspberry Pi application to enable people to get exercise while programming in B4
    \subsubsection*{Daily report bot written in Rust}
      \faicon{github} \href{https://github.com/diohabara/honjitsu}{https://github.com/diohabara/honjitsu}
      \\Created a bot that collects personal information and creates a daily report
    \subsubsection*{A toy C compiler written in C itself}
      \faicon{github} \href{https://github.com/diohabara/ccc}{https://github.com/diohabara/ccc}
      \\Implemented a toy C compiler in C with a lexer, parser, and code generator.
      It generates x86-64 code.
    \subsubsection*{NSA Codebreaker Challenge 2022}
      Currently engaged in this CTF-style security contest.

  \subsection*{Team Project}
    \subsubsection*{The top 3 in ICCAD 2021 CAD Contest}
      Teamed up with other lab members and worked on the circuit generation section in B4.
      Wrote evaluations of methods based on our criteria, actual statistics, and goal.
      My main task was testing what combination of methods we should choose.
  \end{multicols}
\end{document}