\documentclass[10pt]{article}
\usepackage[T1]{fontenc}
\usepackage{newpxtext, newpxmath}
\usepackage{dirtytalk}
\usepackage[margin=15mm]{geometry}
\usepackage{fancyhdr}
\usepackage{cmbright} % font
\usepackage{hyperref}
% configuration
% hyperlink coloring
\hypersetup{
    colorlinks=true,
    filecolor=magenta,
    urlcolor=cyan,
}
\urlstyle{same}
% set header
\pagestyle{fancy}
\lhead{
  Curriculum Vitae/Resume,
  Takemaru Kadoi,
  \href{https://github.com/diohabara/}{GitHub}
  , \href{https://www.linkedin.com/in/takemaru-kadoi}{LinkedIn}
  , \href{mailto:diohabara@gmail.com}{diohabara@gmail.com}
}
% macro

% content
\title{\vspace{-1cm}Curriculum Vitae/Resume}
\author{Takemaru Kadoi}
\date{}

\begin{document}

\section*{Educations}
  \subsection*{The University of Texas at Dallas, August 2022 - Present}
    Graduate Student in Computer Science, Erik Jonsson School of Engineering and Computer Science, ECSS
  \subsection*{The University of Tokyo, April 2017 - March 2022}
    Bachelor in EECS(Electrical Electronics and Computer Science), Department of Engineering
    \\
    Wrote a senior thesis, "Type- and Sequential Effect-Guided Programming by Example," supervised by \href{https://www.cad.t.u-tokyo.ac.jp/en/}{Masahiro FUJITA}
  \subsection*{Stanford University, June 2019 - September 2019}
    Exchange Student in Computer Science
    \\
    Took courses of "CS 161 Design and Analysis of Algorithms" and "CS 103 Mathematical Foundations of Computing."

\section*{Employments}
  \subsection*{\href{https://www.indeed.com/about}{Indeed}, Software Engineer Intern, July 2022 - August 2022}
    Created a proof-of-concept data pipeline for its data infrastructure so that the data team and other teams could easily collaborate with the same API in different programming languages.
    The implementation allowed more people to analyze data in the company.
  \subsection*{\href{https://hatenacorp.jp/}{Hatena}, Software Engineer Intern, February 2022 - May 2022}
    Migrated CI/CD pipelines from Jenkins to GitHub Actions and sped them up, resulting in better production quality by improving the development cycle and increasing the number of reviews and deployments.
  \subsection*{\href{https://q-squared.jp}{Q-Squared}, Software Engineer Intern, June 2021 - February 2022}
    Created a proof-of-concept for network acceleration.
    My work contributed to the decision about what the company would select as a network library for the company.
  \subsection*{\href{https://about.mercari.com/en}{Mercari}, Software Engineer Intern, August 2020 - September 2020}
    Created a static analysis tool for the Go programming language.
  \subsection*{\href{https://www.flywheel.jp}{FLYWHEEL}, Software Engineer Intern, July 2020 - August 2020}
    Introduced the speed layer of the \href{https://www.databricks.com/glossary/lambda-architecture}{lambda architecture} in the corporate data platform.
    The new layer potentially expanded the corporate business because streaming data analytics was useful for real-time analysis.
    \\
    \href{https://www.flywheel.jp/topics/20200917}{Related Blog post in Japanese}
  \subsection*{\href{https://wantedlyinc.com/ja}{Wantedly}, Software Engineer Intern, August 2019 - September 2019}
    Improved existing functions of \href{https://people.wantedly.com/}{Wantedly People}, an HR application, and added new functionalities.
    Those new functionalities were related to user management and enabled the company to find malicious users.

\section*{Projects}
  \subsection*{A toy C compiler written in C: \href{https://github.com/diohabara/ccc}{ccc}}
    Wrote a toy C compiler with an original lexer, parser, and code generator as hobby.
  \subsection*{The top 3 in \href{http://iccad-contest.org/2021/}{ICCAD 2021 CAD Contest}}
    Teamed up with other B4 lab members and worked on the circuit generation section.
    My main task was testing what combinations of libraries we should choose, and I wrote objective evaluations of them based on our criteria, actual statistics, and goal.
  \subsection*{BrainF**k interpreter on Raspberry Pi written in Python: \href{https://github.com/diohabara/muscle_fuck}{muscle\_f**k}}
    Created a Raspberry Pi app to enable people to get exercise while programming in B4
  \subsection*{Implementation of TCP/IP protocol with \href{https://www.klab.com/en/}{KLab}}
    Implemented TCP/IP protocol in C with mentors and other participants as an internship project in B4.
  \subsection*{Addition of some functionalities to \href{https://slides.com/diohabara/deck}{Firefox}}
    Added additional bookmark properties to Firefox as a school project in B3.

\end{document}
