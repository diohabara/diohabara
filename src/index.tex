\documentclass[10pt]{article}
\usepackage[T1]{fontenc}
\usepackage{newpxtext, newpxmath}
\usepackage{dirtytalk}
\usepackage[margin=15mm]{geometry}
\usepackage{fancyhdr}
\usepackage{cmbright} % font
\usepackage{hyperref}
% configuration
% hyperlink coloring
\hypersetup{
    colorlinks=true,
    filecolor=magenta,
    urlcolor=cyan,
}
\urlstyle{same}
% set header
\pagestyle{fancy}
\lhead{
  Curriculum Vitae/Resume,
  Takemaru Kadoi,
  \href{https://github.com/diohabara/}{GitHub}
  , \href{https://www.linkedin.com/in/takemaru-kadoi}{LinkedIn}
  , \href{mailto:diohabara@gmail.com}{diohabara@gmail.com}
}
% macro

% content
\title{\vspace{-1cm}Curriculum Vitae/Resume}
\author{Takemaru Kadoi}
\date{}

\begin{document}

\section*{Educations}
  \subsection*{The University of Texas at Dallas, August 2022 - Present}
    Master Student in Computer Science, Erik Jonsson School of Engineering and Computer Science
  \subsection*{The University of Tokyo, April 2017 - March 2022}
    Bachelor in EECS(Electrical Electronics and Computer Science), Department of Engineering
    \\
    Studied in Stanford University as an exchange student in Computer Science
    \\
    Wrote a senior thesis, "Type- and Sequential Effect-Guided Programming by Example," supervised by \href{https://scholar.google.com/citations?user=tYabznkAAAAJ}{Prof Masahiro FUJITA}

\section*{Employments}
  \subsection*{\underline{Software Engineer Intern}, Indeed, 06/2022 - 08/2022 \hfill Tokyo, Japan}
    Created a proof-of-concept data pipeline for its data infrastructure so that the data team and other teams could easily collaborate with the same API in different programming languages.
    The implementation allowed more people to analyze data in the company.
  \subsection*{\underline{Software Engineer Intern}, Indeed, 02/2022 - 05/2022 \hfill Tokyo, Japan}
    Migrated CI/CD pipelines from Jenkins to GitHub Actions and sped them up, resulting in better production quality by improving the development cycle and increasing the number of reviews and deployments.
  \subsection*{\underline{Software Engineer Intern}, Q-Squared, 06/2021 - 02/2022 \hfill Tokyo, Japan}
    Created a proof-of-concept for network acceleration.
    My work contributed to the decision about what the company would select as a network library for the company.
  \subsection*{\underline{Software Engineer Intern}, Mercari, 08/2020 - 09/2020 \hfill Tokyo, Japan}
    Created a static analysis tool for the Go programming language.
  \subsection*{\underline{Software Engineer Intern}, FLYWHEEL, 06/2020 - 08/2020 \hfill Tokyo, Japan}
    Introduced the speed layer of the lambda architecture in the corporate data platform.
    The new layer potentially expanded the corporate business because streaming data analytics was useful for real-time analysis.
  \subsection*{\underline{Software Engineer Intern}, Wantedly, 08/2019 - 09/2019 \hfill Tokyo, Japan}
    Improved existing functions of Wantedly People, an HR application, and added new functionalities.
    Those new functionalities were related to user management and enabled the company to find malicious users.

\section*{Projects}
  \subsection*{A toy C compiler written in C: \href{https://github.com/diohabara/ccc}{ccc}}
    Wrote a toy C compiler with an original lexer, parser, and code generator as hobby.
  \subsection*{The top 3 in \href{http://iccad-contest.org/2021/}{ICCAD 2021 CAD Contest}}
    Teamed up with other lab members and worked on the circuit generation section in B4.
    My main task was testing what combination of libraries we should choose, and I wrote evaluations of them based on our criteria, actual statistics, and goal.
  \subsection*{BrainF**k interpreter on Raspberry Pi written in Python: \href{https://github.com/diohabara/muscle_fuck}{muscle\_f**k}}
    Created a Raspberry Pi app to enable people to get exercise while programming in B4
  \subsection*{Implementation of TCP/IP protocol with KLab}
    Implemented TCP/IP protocol in C with mentors and other participants as an internship project in B4.
  \subsection*{Addition of some functionalities to Firefox}
    Added additional bookmark properties to Firefox as a school project in B3.

\end{document}
